\documentclass[a4paper,10pt,twoside]{article}

% Packages
\usepackage{ucs}
\usepackage[utf8x]{inputenc}
\usepackage{amsmath}
\usepackage{amsfonts}
\usepackage[francais]{babel}
\usepackage{fontenc}
\usepackage{graphicx}
\usepackage{hyperref}
\usepackage[top=2 cm, bottom=2 cm, left=2 cm, right=2 cm]{geometry}
\renewcommand{\FrenchLabelItem}{\textbullet}
% Amélioration des environnements enumerate
\usepackage{enumerate}
\usepackage[urw-garamond]{mathdesign}
\usepackage{concmath}
\usepackage{commath}

\newcommand{\HRule}{\rule{\linewidth}{0.35mm}}
\setlength{\fboxrule}{.5mm}

\title{\HRule\\
MED08: Python \ldots \textit{pour les scientifiques} \\
1-3 février 2016 \\
École doctorale SISEO \\
\href{mailto:ludovic.charleux@univ-smb.fr}{ludovic.charleux@univ-smb.fr}
\HRule
}

\date{}

\begin{document}

\maketitle

\section{Quelques généralités sur Python}


Le langage Python présente plusieurs avantage qui en font un bon outil dans une optique scientifique:
\begin{itemize}
\item C'est un langage généraliste présent dans de nombreuses domaines: calcul scientifique, web, bases de données, jeu vidéo, graphisme, \textit{etc}. C'est un outil polyvalent qu'un chercheur qui ne souhaite pas nécessairement maîtriser plusieurs langages  pourra utiliser pour toutes les tâches numériques dont il a besoin. 
\item Il est présent sur la majorité des plateformes: Windows, Mac OS, Linux, Unix, Android \ldots
\item C'est un langage libre (et donc gratuit) avec un grande communauté d'utilisateurs. Au delà des considérations philosophiques sur le logiciel libre,  ce point est intéressant car il garantit que vos productions sont donc échangeables et distribuables sans contraintes et sans restrictions dans le temps.
\item Il dispose d'un écosystème de libraire et d'outils associés très fourni qui en font un outil qui peut aller bien au delà de la résolution de vos problèmes.
\item La communauté de Python est très vaste ce qui implique que vos questions trouveront quasiment toujours des réponses sur internet mais aussi très peu de bugs sont présents dans Python.

\end{itemize}

\section{Proposition de programme}

Ce programme est indicatif dans la mesure ou il est souhaitable de l'adapter au niveau des participants et à leurs attentes vis-à-vis de l'outil présenté.
\begin{description}
\item[Lundi matin]: 
\begin{itemize}
\item finalisation des installations, 
\item point sur vos attentes,
\item prise en main des outils,
\end{itemize}

\item[Lundi après-midi]: 
\begin{itemize}
\item Programmation de base en Python, 
\item intérêts des bibliothèques scientifiques et graphiques pour le scientifique Numpy, Scipy et Matplotlib. 
\item Exemples divers.
\end{itemize}

\item[Mardi matin]: 
\begin{itemize}
\item Création d'une librairie documentée avec Sphinx en mode collaboratif avec Git et GitHub.
\end{itemize}

\item[Lundi après-midi]: 
\begin{itemize}
\item Thématiques à la carte. Exemples : graphiques avancés pour les publications avec Matplotlib (graphiques complexes, couplages avec Latex), gestion de bases de données avec SQLAlchemy, ... 
\end{itemize}
\item[Mercredi matin]:
\begin{itemize}
\item ttt
\end{itemize}
\item[Mercredi après-midi]: 
\begin{itemize}
\item réalisation d'un projet : chaque participant se propose de résoudre un problème avec les méthodes proposées. 
\item À la fin de cette séance, le projet est réalisé et rendu sous forme de librairie ou de dépôt GITHub.
\end{itemize}
\end{description}


\end{document}
